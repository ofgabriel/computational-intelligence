\documentclass{homework}

\usepackage{tikz}
\usepackage{listings}
\usepackage{amsmath}
\usepackage{systeme}
\usepackage{enumitem}
\usepackage{xcolor}

\lstset
{ %Formatting for code in appendix
    basicstyle=\footnotesize,
    numbers=left,
    stepnumber=1,
    showstringspaces=false,
    tabsize=1,
    breaklines=true,
    breakatwhitespace=false,
}
\setlength{\parindent}{4em}

\title{Inteligência Computacional --- COC 361\linebreak2021/2 \linebreak\linebreak Trabalho Computacional}
\author{Gabriel de Oliveira da Fonseca, Gustavo Pires Machado}

\begin{document}

\maketitle

\pagebreak

\section{Introdução}

\begin{itemize}
\color{red}
    \item Descrição do problema.
    \item Pesquisa bibliográfica (opcional).
\end{itemize}

Para o presente trabalho, o conjunto de dados escolhido reúne dados coletados entre 1º de Julho de 2015 e 31 de Agosto
de 2017 por uma rede hoteleira que incluem diversos atributos relacionados às reservas efetuadas por seus clientes.
Utilizando-se das diversas metodologias de Inteligência Computacional discutidas ao longo do curso, o trabalho tem por
objetivo a construção de um modelo de classifcação para a previsão de reservas canceladas. A partir deste modelo,
espera-se que a rede hoteleira possa se beneficiar de uma maior previsibilidade das reservas que serão efetivamente
concretizadas, aumentando por fim sua margem de lucro.

\section{Dataset e Tecnologia}

\begin{itemize}
\color{red}
    \item Descrição dos Dados.
    \item Apresentação da tecnologia.
\end{itemize}

O dataset escolhido possui 36 colunas e foi retirado da plataforma Kaggle \cite{kaggle}. Conforme disposto na tabela
abaixo, 20 dessas colunas são numéricas, e portanto 16 categóricas.

** tabela com as colunas.

\section{Metodologia}

\begin{itemize}
    \color{red}
        \item Apresentação da solução do problema proposto.
        \item Descrição teórica (matemática) dos modelos utilizados.
\end{itemize}

\section{Resultados}

\begin{itemize}
    \color{red}
        \item Visualização e Caracterização dos dados (distribuições, correlações, etc.)
        \item Descrição do procedimento de validação (validação cruzada)
        \item Resultados dos modelos lineares:
            \begin{enumerate}
                \item Regressão Linear / Regressão Logística
                \item Classificação Bayesiana (problemas de classificação)
            \end{enumerate}
        \item Resultados dos modelos não lineares
            \begin{enumerate}
                \item Árvores de decisão
                \item Random Forest (testar pelo menos 2 opções de hiperparâmetros)
                \item Gradient Boosting (testar pelo menos 2 opções de hiperparâmetros)
                \item SVM (testar pelo menos 2 opções de hiperparâmetros)
                \item Redes Neurais (testar pelo menos 3 topologias/hiperparâmetros)
            \end{enumerate}
        \item Discussão e comparação dos resultados
\end{itemize}

\section{Conclusões}

\begin{itemize}
    \color{red}
        \item Discussão sobre as características do problema
        \item Discussão dos resultados obtidos em função das características do problema
        \item Recomendação sobre o melhor modelo para a aplicação
        \item Trabalhos futuros (opcional)
\end{itemize}

\begin{thebibliography}{1}
    \bibitem{kaggle} \url{https://www.kaggle.com/mojtaba142/hotel-booking}. Acessado em 20/02/2022.
\end{thebibliography}

\end{document}
